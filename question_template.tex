% ============================================
% TEMPLATE FOR A SINGLE QUESTION
% Use this structure when adding content from Prompt B output
% ============================================

\section{Question X: [Question Title]}
\label{sec:qXX}

\textit{[Subtitle/Description from Substack post]}

% ============================================
% Section 1: The Interview Scenario
% ============================================

You're in a Senior ML Interview at [Company]. The interviewer sets a trap:

"[Interview question]"

[Describe candidate's reasoning and common wrong answer]

% ============================================
% Section 2: The Trap Explained
% ============================================

[Explain why the trap is deceptive, why the wrong answer seems correct, and real-world consequences]

\begin{notebox}
[Important warning about the trap]
\end{notebox}

% ============================================
% Section 3: Technical Deep Dive
% ============================================

\subsection{Mathematical Foundation}

[Mathematical formulations, equations, definitions]

\begin{dfnbox}{[Key Term]}{label}
[Definition of key term]
\end{dfnbox}

\begin{thmbox}{[Theorem/Key Result]}{label}
[Mathematical theorem or key result]
\tcblower
\begin{proof}
[Proof if applicable]
\end{proof}
\end{thmbox}

\subsection{Algorithmic Details}

[Algorithm descriptions, pseudocode, complexity analysis]

\begin{tecbox}{[Technique Name]}{label}
[Description of technique or algorithm]
\end{tecbox}

\begin{codebox}{[Algorithm Name]}{label}
\begin{amzcode}{python}
# Pseudocode or code example
def algorithm():
    pass
\end{amzcode}
\end{codebox}

\subsection{System Architecture}

[System architecture implications, design patterns, infrastructure requirements]

\begin{genbox}{[Architectural Pattern]}
[Description of architectural pattern or design decision]
\end{genbox}

% ============================================
% Section 4: Why It Fails
% ============================================

[Detailed explanation of failure mechanism, edge cases, and concrete examples]

\begin{exbox}{[Concrete Example]}{label}
[Detailed example with specific numbers showing why it fails]
\end{exbox}

[Analysis of why the example fails, with calculations]

% ============================================
% Section 5: The Solution
% ============================================

[Describe the correct approach, key insight, and step-by-step implementation]

\begin{tecbox}{[Solution Technique]}{label}
[Description of the solution technique]
\end{tecbox}

[Implementation details, trade-offs, and considerations]

% ============================================
% Section 6: Why The Solution Works
% ============================================

[Intuitive explanation first, then theoretical foundation]

\begin{thmbox}{[Theoretical Result]}{label}
[Theoretical justification]
\tcblower
\begin{proof}
[Proof if applicable]
\end{proof}
\end{thmbox}

[Empirical evidence, benchmarks, real-world validation]

% ============================================
% Section 7: Related Research Papers
% ============================================

\subsection{Paper: [Paper Title]}

\textbf{Authors:} [Author names], \textbf{Year:} [Year]. \textit{[Paper Title]}. [Venue/ArXiv].

\textbf{Key Contribution:} [2-3 sentence summary of main contribution]

\textbf{Relevance:} [How it relates to this question]

\textbf{Key Technique:} [Specific method or algorithm if applicable]

[Repeat for each paper]

% ============================================
% Section 8: Interview Answer Template
% ============================================

\begin{genbox}{The Answer That Gets You Hired}
[Complete, polished interview answer that demonstrates understanding]
\end{genbox}

\subsection{Key Points to Emphasize}

\begin{itemize}
    \item [Key point 1]
    \item [Key point 2]
    \item [Key point 3]
\end{itemize}

\subsection{Common Follow-up Questions}

\begin{itemize}
    \item \textbf{Q:} [Follow-up question] \textbf{A:} [Answer]
    \item \textbf{Q:} [Follow-up question] \textbf{A:} [Answer]
\end{itemize}

% ============================================
% Section 9: Key Takeaways
% ============================================

\subsection{Summary}

\begin{itemize}
    \item [Key takeaway 1]
    \item [Key takeaway 2]
    \item [Key takeaway 3]
    \item [Key takeaway 4]
    \item [Key takeaway 5]
\end{itemize}

\subsection{Related Concepts}

[Related questions, prerequisites, advanced topics, connections to other interview questions]

% ============================================
% End of Question Template
% ============================================

